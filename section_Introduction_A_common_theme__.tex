\section{Introduction}

A common theme in mental health is to diagnose and categorise based upon observed behaviours and infer severity by evaluating the impact behaviour has on an individual's functioning. While important, behaviours are not the only window to understanding mental disorders. Understanding the underlying mechanisms driving behaviour abnormalities has been a chief goal of research for decades, with notable contributions from fields such as cognition and neuroscience \cite{Frank_2015}. These fields have begun revealing phenotypic overlaps between disorders with dissimilar behavioural presentations. Obsessive-compulsive disorder (OCD) and problem gambling (PG) are two such examples of this, with research suggesting shared links to the inter-related constructs of impulsivity and compulsivity \cite{Tavares_2007} as well as overlapping biomarkers centered on corticostriatal circuits \cite{van_Holst_2010,Harrison_2009,Harrison_2013}. Despite this, in the primary diagnostic manuals for mental health (DSM-5), emphasis remains on the nature and the impact of behaviours. Here we show that regrouping an OCD cohort and a PG cohort based upon questionnaire measures that tap broadly into impulsivity and compulsivity reveals new insights into the underlying neural mechanisms linked to these disorders.

I'm a dick.

% Paragraph demonstrating the rationale of using OCD and PG: links to imp/comp

PG and OCD are associated with distinct patterns of behaviour and are typically characterised by issues with impulsivity \cite{Goudriaan_2004} and compulsivity \cite{Stein_1994}, respectively. These constructs were traditionally viewed at opposite ends of a single dimension \cite{Fineberg_2009}, however, literature is now converging on the idea that impulsivity and compulsivity are orthogonally related and can overlap \cite{Fineberg_2009,Fontenelle_2011}. For instance, \citet{Tavares_2007} supported this by demonstrating that impulsive and compulsive aspects of personality were found in both PG and OCD.

% Paragraph demonstrating the rationale of using OCD and PG: links to circuitry

Converging evidence from neuroimaging investigations has identified that these disorders may also share common neural underpinnings centred on abnormal corticostriatal circuits. Moreover, these circuits have a known association with impulsivity and compulsivity \cite{Torregrossa2008}.

% Paragraph defining the idea of "phenotyping". Scrubbing out diagnostic boundaries and regrouping based on common expressions in constructs (imp comp in my case)




