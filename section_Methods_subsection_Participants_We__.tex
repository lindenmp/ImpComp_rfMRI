\section{Methods}

\subsection{Participants}

We recruited 20 patients with OCD, 20 patients with PG, and 20 healthy controls (HCs). All participants were unrelated and matched on gender, age, and IQ. Sample demographics are presented in Table X. All participants provided informed written consent in accordance with Monash University ethics committee guidelines.

\subsection{Recruitment and eligibility}

OCD patients were recruited from specialist clinical services located in Melbourne, Australia. PG patients and HCs were recruited from the community. To be eligible for study inclusion, all participants were required to have no lifetime history of concussion, neurological disease, or cannabis/alcohol dependence. OCD patients were required to score >8 on the severity section of the Florida Obsessive-Compulsive Inventory (FOCI: XXX) and have their diagnosis confirmed by treatment services and the Mini International Neuropsychiatric Interview version 5 (MINI: XXX). All PG patients engaged at least weekly in Electronic Gaming Machine (EGM) gambling, were required to score >8 on the Problem Gambling Severity Index (PGSI: XXX), and had their diagnosis confirmed by the Structured Clinical Interview for DSM-IV for PG (SCID-PG: XXX). The presence of either depression or anxiety, indexed by the MINI, in either OCD or PG patients was not excluded so long as the OCD and PG symptoms constituted the primary cause of distress and interference in the patient’s life. However, all other psychiatric disorders were excluded, including the concurrent presence of OCD and PG within a single patient.

\section{Measures}

\subsection{Phenotyping}
	
The following questionnaires were used to proxy the impulsive-compulsive phenotype outlined above. The criteria for these measures were two fold. First, they needed to link conceptually to issues with either impulsivity or compulsivity that underlie PG and OCD. Second, they needed to be generalizable such that they did not tap into disorder-specific behaviours like chasing losses in PG or classical subtypes of OCD such as washing or checking.

\subsubsection{Behavioral Inhibition and Behavioral Activation Sclaes (BIS/BAS)} 
The BIS/BAS (Carver & White, 1994) is 24 item scale that measures to what extent an individual’s behaviour is controlled by inhibitory or appetitive motivations. The BAS consists of three subscales: (i) reward responsiveness (e.g. “When I get something I want, I feel excited and energized”), (ii) drive (e.g. “When I want something, I usually go all-out to get it”, and (iii) fun seeking (e.g. “I often act on the spur of the moment”).

\subsubsection{Impulsive Behaviour Scale (UPPS-P)}
The UPPS-P (Cyders et al., 2007) is a 59 item scale which is a revised version of the original UPPS (Whiteside, Lynam, Miller, & Reynolds, 2005).  The UPPS-P has five subscales: (i) negative urgency, (ii) positive urgency, (ii) lack of premeditation, (iv) lack of perseverance, and (v) sensation seeking. Items are scored on a 4-point scale ranging from 1 (agree strongly) to 4 (disagree strongly). Good validity and reliability has been reported (Cyders et al., 2007; Whiteside et al., 2005).

\subsubsection{Intolerance of Uncertainty Scale (IUS-12)}
The IUS-12 is a short version of the original 27 item scale that measures responses to uncertainty and ambiguity (Freeston, Rhéaume, Letarte, Dugas, & Ladouceur, 1994).  The 12 items are rated on a 5-point Likert scale ranging from 1 (not characteristic of me) to 5 (entirely characteristic of me). The 12-item version consists of two subscales. The subscales are (i) prospective IU, consisting of 7 items (e.g. “I can’t stand being taken by surprise”) and (ii) inhibitory IU, consisting of 5 items (e.g. “When it’s time to act, uncertainty paralyses me”).  Strong reliability and validity has been reported (Carleton, Norton, & Asmundson, 2007).

\subsubsection{Obsessive Beliefs Questionnaire (OBQ-44)}
The OBQ-44 is a short version of the original 87 item scale that consists of 44 belief statements considered characteristic of obsessive thinking (Obsessive Compulsive Cognitions Working Group, 2005). There are four subscales that represent key belief domains often distorted in compulsive individuals. The subscales are (i) responsibility (ii) threat estimation, (iii) perfectionism/certainty, and (iv) importance/control of thoughts (Myers, Fisher, & Wells, 2008). Participants indicate to what extent they believe the items are characteristic of themselves using a 7-point rating scale that ranges from (-3) “disagree very much” to (+3) “agree very much”.


